\chapter{AWS setup}
\label{ch:aws-setup}

The labs in this course will use Amazon Web Services, or AWS.
To do the labs in this course, you will need your own AWS account.

Note that this is not a philosophical endorsement of Amazon.
There are other cloud providers too - IBM, Google, Microsoft Azure, others.
Most of the concepts encountered in AWS translate to the others.

\section{Charges}
\label{sec:charges}

Almost all AWS services are chargeable.
Many services have a time-limited free tier.
We will stay almost entirely within the free tier. 

You will need a credit / debit card to sign up for AWS.
If you don't have one you should be able to use a prepaid card or use an online card like Revolut.
\textit{I have not tested this option.}

\section{Sign up for an AWS account}
\label{sec:signup}

If you already have an AWS account you should skip this section.

Sign up for an AWS account by visiting the link:\\
\url{https://portal.aws.amazon.com/billing/signup}

There is a 5-step process to signing up.

For the account type you should choose Personal.

\section{Log in to the console}

Make sure that you can login to the AWS console using the credentials from the \autoref{sec:signup}.
The AWS console can be accessed at:\\
\url{https://aws.amazon.com/console/}

Bookmark the AWS console link in your Browser. You will need it often.


\subsection{Region setup}
\label{sec:region-setup}

AWS is divided into a number of regions.

On the top right of the AWS Console, it will probably say \textit{N Virginia}.
Click this and change it to Ireland (eu-west-1).
The college firewall only allows some connections we need for our work in this module to AWS resources in the eu-west-1 region.

We will look at regions again as part of our study of Global Infrastructure.

