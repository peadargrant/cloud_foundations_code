  
% \section{Setup AWS command-line
% interface}\label{setup-aws-command-line-interface}

% The AWS Command Line Interface is a client program that runs on your
% local PC to allow you to manage AWS resources from the command-line
% (PowerShell, Bash). We will use the command-line extensively in this
% module.

% \subsection{PowerShell execution
% policy}\label{powershell-execution-policy}

% PowerShell by default will not allow scripts to run that were downloaded
% from online. The following command will change this behaviour:

% \begin{verbatim}
% Set-ExecutionPolicy -ExecutionPolicy RemoteSigned -Scope CurrentUser
% \end{verbatim}

% \subsection{AWS CLI installation}\label{aws-cli-installation}

% If you're on a lab computer OR if you already have the AWS CLI
% installed, then skip ahead to .

% Install the command-line tools from \url{https://aws.amazon.com/cli/}

% \subsection{Config file setup}\label{sec:config-file-setup}

% There is a script file \texttt{setup\_config\_file.ps1} that will setup
% your configuration file for you. Run it once.

% \subsection{Access key setup}\label{sec:access-key-setup}

% \textbf{Needs to be done EVERY time you log in on a student account!}

% Log in to AWS academy. Go to
% \url{https://awsacademy.instructure.com/courses/8294/modules/items/794040}.

% Hit the \emph{AWS Details} button. Look for \emph{Cloud Access} and
% \emph{AWS CLI} on the right. Click \emph{Show}. Copy this.

% \subsubsection{PowerShell}\label{powershell}

% You can paste the above using:

% \begin{verbatim}
% Get-Clipboard | Out-File ~/.aws/credentials
% \end{verbatim}

% Alternatively you can use the script:

% \begin{verbatim}
% .\paste_credentials.ps1
% \end{verbatim}

% \subsubsection{Manual alternative}\label{manual-alternative}

% Paste the contents into a file named EXACTLY

% \begin{verbatim}
% C:\Users\yourusername\.aws\credentials
% \end{verbatim}

% (no \texttt{.txt} etc at the end).

% \subsection{Check CLI configured}\label{check-cli-configured}

% To check that the AWS CLI is correctly configured, you can try running
% the command:

% \begin{verbatim}
% aws ec2 describe-instances
% \end{verbatim}

% If it shows something similar to:

% \begin{verbatim}
% {
%     "Reservations": []
% }
% \end{verbatim}

% then the AWS CLI is working OK.

% \section{Mac, Linux, UNIX users}\label{mac-linux-unix-users}

% \emph{Windows users can ignore this section.}

% Mac, Linux, Unix users will have no problems installing the AWS CLI, and
% the commands work identically to those on Windows.

% The difference between Mac/Linux and Windows centres on the use of
% Bash/zsh by Mac/Linux/UNIX vs PowerShell on Windows. The AWS CLI is
% perfectly scriptable using Bash, particularly in conjunction with
% \texttt{jq} to parse JSON. However, some of the scripts you will be
% provided with in this module will be PowerShell only due to time
% constraints.

% The good news is that PowerShell Core 7 can be installed easily on a Mac
% with no issues. You \emph{do not} need a Windows VM on your Mac to use
% any of the PowerShell or AWS commands in \emph{this} course. Please go
% to the \href{https://github.com/PowerShell/PowerShell}{PowerShell page
% on GitHub} for instructions.

% When you have PowerShell installed, open the Terminal app and type
% \texttt{pwsh} and you'll be at a PowerShell prompt. Repeat the
% \texttt{aws\ ec2\ describe-instances} command to confirm that the
% \texttt{aws} command is available in PowerShell.

% \section{Check script}\label{check-script}

% Run the \texttt{lab\_checks.ps1} powershell script to confirm that your
% environment is set up correctly.
