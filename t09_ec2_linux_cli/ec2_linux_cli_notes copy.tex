\section{Creating a VPC}\label{creating-a-vpc}

Before creating any resources on AWS, you should draw out your VPC as
best you can on paper. When creating VPCs and other resources, try to be
systematic with your naming!

We will work through an example of a VPC (named \texttt{LAB\_VPC}) with
the CIDR block 10.0.0.0/16 with the 10.0.1.0/24 address range assigned
to a single subnet \texttt{LAB\_1\_SN}. This VPC will be setup so that
it connects to the public internet.

\begin{itemize}
\item
  VPC 10.0.0.0/16 named \texttt{LAB\_VPC}
\item
  Subnet 10.0.1.0/24 \texttt{LAB\_1\_SN}

  \begin{itemize}
  \tightlist
  \item
    Auto-assign public IP is turned on. This means that EC2 instances
    launched in this subnet will also receive a public IP that is
    transparently routed to their private IP by AWS using NAT.
  \end{itemize}
\item
  Internet gateway \texttt{LAB\_IGW} attached to VPC
\item
  Route table (named \texttt{LAB\_RT}) with routes for:

  \begin{itemize}
  \item
    Local traffic 10.0.0.0/16 sent locally
  \item
    All traffic 0.0.0.0/0 routed via \texttt{LAB\_IGW}
  \end{itemize}
\end{itemize}

These notes show how to create a VPC using the CLI. The first thing to
remember is to use the \texttt{help} subcommand of \texttt{aws}
liberally. Some commands have required arguments (things that must
specify) and will tell you if you've omitted them. To see what effect
the commands are having you can follow along at the same time in the
console and see the effect of your changes.

\subsection{VPC creation}\label{vpc-creation}

To create a VPC using the CIDR block 10.0.0.0/16 we can say:

\begin{verbatim}
aws ec2 create-vpc --cidr-block 10.0.0.0/16
\end{verbatim}

This will return a JSON-formatted response similar to:

\begin{Shaded}
\begin{Highlighting}[]
\FunctionTok{\{}
    \DataTypeTok{"Vpc"}\FunctionTok{:} \FunctionTok{\{}
        \DataTypeTok{"CidrBlock"}\FunctionTok{:} \StringTok{"10.0.0.0/16"}\FunctionTok{,}
        \DataTypeTok{"DhcpOptionsId"}\FunctionTok{:} \StringTok{"dopt-71d01c16"}\FunctionTok{,}
        \DataTypeTok{"State"}\FunctionTok{:} \StringTok{"pending"}\FunctionTok{,}
        \DataTypeTok{"VpcId"}\FunctionTok{:} \StringTok{"vpc-0afdc142d97b1eaaa"}\FunctionTok{,}
        \DataTypeTok{"OwnerId"}\FunctionTok{:} \StringTok{"637116340434"}\FunctionTok{,}
        \DataTypeTok{"InstanceTenancy"}\FunctionTok{:} \StringTok{"default"}\FunctionTok{,}
        \DataTypeTok{"Ipv6CidrBlockAssociationSet"}\FunctionTok{:} \OtherTok{[]}\FunctionTok{,}
        \DataTypeTok{"CidrBlockAssociationSet"}\FunctionTok{:} \OtherTok{[}
            \FunctionTok{\{}
                \DataTypeTok{"AssociationId"}\FunctionTok{:} \StringTok{"vpc-cidr-assoc-0d6ab9228eec46921"}\FunctionTok{,}
                \DataTypeTok{"CidrBlock"}\FunctionTok{:} \StringTok{"10.0.0.0/16"}\FunctionTok{,}
                \DataTypeTok{"CidrBlockState"}\FunctionTok{:} \FunctionTok{\{}
                    \DataTypeTok{"State"}\FunctionTok{:} \StringTok{"associated"}
                \FunctionTok{\}}
            \FunctionTok{\}}
        \OtherTok{]}\FunctionTok{,}
        \DataTypeTok{"IsDefault"}\FunctionTok{:} \KeywordTok{false}
    \FunctionTok{\}}
\FunctionTok{\}}
\end{Highlighting}
\end{Shaded}

A few things to note about the response:

\begin{itemize}
\item
  Most AWS commands will return output formatted as JSON.
\item
  The VPC ID is assigned by AWS, not us.
\end{itemize}

If you look at the console following creation of your VPC you'll see it
listed if you hit refresh.

To name our VPC as \texttt{LAB\_VPC} we issue the command:

\begin{verbatim}
aws ec2 create-tags --resources vpc-0afdc142d97b1eaaa --tags Key=Name,Value=LAB_VPC
\end{verbatim}

Obviously you'll have to change the VPC ID in the above command to your
own. Hitting refresh in the console should show the name now appearing.

\subsection{Subnet creation}\label{subnet-creation}

Creating a subnet requires two pieces of information: the VPC ID and the
CIDR block (which must be a subset of the VPC CIDR block). For a simple
VPC with one subnet, the subnet's CIDR block can be the same as the
VPC's CIDR block. But normally it's good to have it smaller (giving us
the option later to add another subnet). Here we will create a single
subnet with the CIDR block 10.0.0.0/24 inside the VPC (CIDR block
10.0.0.0/16).

\begin{verbatim}
aws ec2 create-subnet --vpc-id vpc-0afdc142d97b1eaaa --cidr-block 10.0.0.0/24
\end{verbatim}

Just like the VPC, the subnet creation will return:

\begin{Shaded}
\begin{Highlighting}[]
\FunctionTok{\{}
    \DataTypeTok{"Subnet"}\FunctionTok{:} \FunctionTok{\{}
        \DataTypeTok{"AvailabilityZone"}\FunctionTok{:} \StringTok{"eu-west-1b"}\FunctionTok{,}
        \DataTypeTok{"AvailabilityZoneId"}\FunctionTok{:} \StringTok{"euw1-az3"}\FunctionTok{,}
        \DataTypeTok{"AvailableIpAddressCount"}\FunctionTok{:} \DecValTok{65531}\FunctionTok{,}
        \DataTypeTok{"CidrBlock"}\FunctionTok{:} \StringTok{"10.0.0.0/24"}\FunctionTok{,}
        \DataTypeTok{"DefaultForAz"}\FunctionTok{:} \KeywordTok{false}\FunctionTok{,}
        \DataTypeTok{"MapPublicIpOnLaunch"}\FunctionTok{:} \KeywordTok{false}\FunctionTok{,}
        \DataTypeTok{"State"}\FunctionTok{:} \StringTok{"available"}\FunctionTok{,}
        \DataTypeTok{"SubnetId"}\FunctionTok{:} \StringTok{"subnet-0530a2180c3f71e62"}\FunctionTok{,}
        \DataTypeTok{"VpcId"}\FunctionTok{:} \StringTok{"vpc-0afdc142d97b1eaaa"}\FunctionTok{,}
        \DataTypeTok{"OwnerId"}\FunctionTok{:} \StringTok{"637116340434"}\FunctionTok{,}
        \DataTypeTok{"AssignIpv6AddressOnCreation"}\FunctionTok{:} \KeywordTok{false}\FunctionTok{,}
        \DataTypeTok{"Ipv6CidrBlockAssociationSet"}\FunctionTok{:} \OtherTok{[]}\FunctionTok{,}
        \DataTypeTok{"SubnetArn"}\FunctionTok{:} \StringTok{"arn:aws:ec2:eu-west-1:637116340434:subnet/subnet-0530a2180c3f71e62"}
    \FunctionTok{\}}
\FunctionTok{\}}
\end{Highlighting}
\end{Shaded}

A few interesting things to note about the response:

\begin{itemize}
\item
  AWS has assigned a \texttt{SubnetId} for us.
\item
  Each subnet is actually linked to a physical availability zone, here
  \texttt{eu-west-1b} (within the \texttt{eu-west-1} region).
\item
  Note also how \texttt{eu-west-1b} is actually known as
  \texttt{euw1-az3}. This is because the \texttt{a,\ b,\ c} availability
  zones are actually rotated between different accounts to balance load.
  Your \texttt{eu-west-1b} might be another person's
  \texttt{eu-west-1c}.
\end{itemize}

We can name the subnet the same way as the VPC:

\begin{verbatim}
aws ec2 create-tags --resources subnet-0530a2180c3f71e62 --tags Key=Name,Value=LAB_SN
\end{verbatim}

\subsection{Internet gateway}\label{internet-gateway}

An internet gateway needs to be created and then attached to the correct
VPC. To create an internet gateway we type:

\begin{verbatim}
aws ec2 create-internet-gateway
\end{verbatim}

with JSON response like:

\begin{Shaded}
\begin{Highlighting}[]
\FunctionTok{\{}
    \DataTypeTok{"InternetGateway"}\FunctionTok{:} \FunctionTok{\{}
        \DataTypeTok{"Attachments"}\FunctionTok{:} \OtherTok{[]}\FunctionTok{,}
        \DataTypeTok{"InternetGatewayId"}\FunctionTok{:} \StringTok{"igw-01fe2befea2cd8a27"}\FunctionTok{,}
        \DataTypeTok{"OwnerId"}\FunctionTok{:} \StringTok{"637116340434"}\FunctionTok{,}
        \DataTypeTok{"Tags"}\FunctionTok{:} \OtherTok{[]}
    \FunctionTok{\}}
\FunctionTok{\}}
\end{Highlighting}
\end{Shaded}

We can name the internet gateway the same as before:

\begin{verbatim}
aws ec2 create-tags --resources igw-01fe2befea2cd8a27 --tags Key=Name,Value='LAB_IGW'
\end{verbatim}

Finally we can attach the internet gateway to a VPC. The
\texttt{aws\ ec2\ attach-internet-gateway} command needs two IDs: the
internet gateway and the VPC.

\begin{verbatim}
aws ec2 attach-internet-gateway `
--internet-gateway-id igw-01fe2befea2cd8a27  `
--vpc-id vpc-0afdc142d97b1eaaa
\end{verbatim}

This won't return any output unless there's an error.

\subsection{Route tables}\label{route-tables}


In the CLI, we can get a list of all route tables using:

\begin{verbatim}
aws ec2 describe-route-tables
\end{verbatim}

which will show us all the route tables. To ensure we see only the route
table(s) for our VPC we should add a filter to the command (as described
in the help):

\begin{verbatim}
aws ec2 describe-route-tables `
--filters Name=vpc-id,Values=vpc-0afdc142d97b1eaaa
\end{verbatim}

This will give us the route table in JSON:

\begin{Shaded}
\begin{Highlighting}[]
\FunctionTok{\{}
    \DataTypeTok{"RouteTables"}\FunctionTok{:} \OtherTok{[}
        \FunctionTok{\{}
            \DataTypeTok{"Associations"}\FunctionTok{:} \OtherTok{[}
                \FunctionTok{\{}
                    \DataTypeTok{"Main"}\FunctionTok{:} \KeywordTok{true}\FunctionTok{,}
                    \DataTypeTok{"RouteTableAssociationId"}\FunctionTok{:} \StringTok{"rtbassoc-08fd06faf732ee3ca"}\FunctionTok{,}
                    \DataTypeTok{"RouteTableId"}\FunctionTok{:} \StringTok{"rtb-0a895efdab0ad7591"}\FunctionTok{,}
                    \DataTypeTok{"AssociationState"}\FunctionTok{:} \FunctionTok{\{}
                        \DataTypeTok{"State"}\FunctionTok{:} \StringTok{"associated"}
                    \FunctionTok{\}}
                \FunctionTok{\}}
            \OtherTok{]}\FunctionTok{,}
            \DataTypeTok{"PropagatingVgws"}\FunctionTok{:} \OtherTok{[]}\FunctionTok{,}
            \DataTypeTok{"RouteTableId"}\FunctionTok{:} \StringTok{"rtb-0a895efdab0ad7591"}\FunctionTok{,}
            \DataTypeTok{"Routes"}\FunctionTok{:} \OtherTok{[}
                \FunctionTok{\{}
                    \DataTypeTok{"DestinationCidrBlock"}\FunctionTok{:} \StringTok{"10.0.0.0/16"}\FunctionTok{,}
                    \DataTypeTok{"GatewayId"}\FunctionTok{:} \StringTok{"local"}\FunctionTok{,}
                    \DataTypeTok{"Origin"}\FunctionTok{:} \StringTok{"CreateRouteTable"}\FunctionTok{,}
                    \DataTypeTok{"State"}\FunctionTok{:} \StringTok{"active"}
                \FunctionTok{\}}
            \OtherTok{]}\FunctionTok{,}
            \DataTypeTok{"Tags"}\FunctionTok{:} \OtherTok{[]}\FunctionTok{,}
            \DataTypeTok{"VpcId"}\FunctionTok{:} \StringTok{"vpc-0afdc142d97b1eaaa"}\FunctionTok{,}
            \DataTypeTok{"OwnerId"}\FunctionTok{:} \StringTok{"637116340434"}
        \FunctionTok{\}}
    \OtherTok{]}
\FunctionTok{\}}
\end{Highlighting}
\end{Shaded}

First let's name the main route table:

\begin{verbatim}
aws ec2 create-tags `
--resources rtb-0a895efdab0ad7591 `
--tags Key=Name,Value=LAB_RTB
\end{verbatim}

Looking at the JSON output, the Routes list contains one entry. This
routes all traffic to 10.0.0.0/16 addresses locally. We need to add a
route for traffic elsewhere (0.0.0.0/0) to go through the internet
gateway.

\begin{verbatim}
aws ec2 create-route `
--route-table-id rtb-0a895efdab0ad7591 `
--destination-cidr-block 0.0.0.0/0 `
--gateway-id igw-01fe2befea2cd8a27
\end{verbatim}

Re-running the describe route table command should now show two routes.

\section{Security groups}\label{security-groups}

Security groups are essentially a firewall controlling what traffic is
allowed into or out of each instance. For a good description of security
groups type:

\begin{verbatim}
aws ec2 create-security-group help
\end{verbatim}

Each instance may have one or more security groups attached.

Every instance created can have a default security group attached but
this leads to a few problems:

\begin{itemize}
\item
  Hard to get an overview of allowed/denied traffic to instances
  (security risk)
\item
  Hard to reconfigure allowed/denied traffic to a number of instances
  (time consuming, nuisance)
\end{itemize}

Instead we will create a security group and attach it as needed to
instances in our VPC. We will create a \texttt{LAB\_SG} security group
to allow in SSH access.

\subsection{Creating security group}\label{sec:creating-security-group}

Security groups are associated with a VPC, so your VPC must be set up
before creating the security group.

\begin{verbatim}
aws ec2 create-security-group `
--group-name 'LAB_SG' `
--description 'Lab security group' `
--vpc-id vpc-049fe28be9e790384
\end{verbatim}

Successful output will look like:

\begin{verbatim}
{
    "GroupId": "sg-08cafd37327645e81"
}
\end{verbatim}

\subsection{Adding ingress rules}\label{adding-ingress-rules}

We now add an ingress rule to our security group to permit inbound TCP
traffic on Port 22 (SSH) using the command:

\begin{verbatim}
aws ec2 authorize-security-group-ingress `
--group-id sg-08cafd37327645e81 `
--protocol tcp `
--port 22 --cidr 0.0.0.0/0
\end{verbatim}

Note we used 0.0.0.0/0 as the source, meaning from anywhere. We can lock
this down to specific IP addresses or IP ranges (e.g.~your ISP). This is
an exercise for another time!

\subsection{Egress rules}\label{egress-rules}

By default, security groups allow egress of all traffic from instances,
so this doesn't need to be set up.

\section{Instance setup}\label{instance-setup}

We will setup an EC2 instance as follows:

\begin{itemize}
\item
  AMI: Amazon Linux
\item
  Configuration: \texttt{t2.micro}
\item
  VPC: \texttt{LAB\_VPC}
\item
  Subnet: \texttt{LAB\_1\_SN}
\item
  Security group: \texttt{LAB\_SG}
\end{itemize}

Using the command:

\begin{verbatim}
aws ec2 run-instances \
 --subnet-id subnet-08f1cd51b749c3ce2 \
--image-id ami-0bb3fad3c0286ebd5 \
--instance-type t2.micro \
--key-name LAB_KEY \
--security-group-ids sg-0ea635cfaa7ab103e
\end{verbatim}

\begin{itemize}
\item
  The \texttt{image-id} is the AMI
\item
  Instance type is \texttt{t2.micro} (good for general purpose stuff)
\item
  The SSH key \texttt{LAB\_KEY} is to be used.
\item
  Security group is the ID of the \texttt{LAB\_SG}.
\end{itemize}

Image IDs are region and account-dependent. They also get updated as
Amazon update the images.

\subsection{Available image names}\label{available-image-names}

We can use a tool called Systems Manager to look up available AMIs:

\begin{verbatim}
# print out list of Linux AMIs
aws ssm get-parameters-by-path `
--path /aws/service/ami-amazon-linux-latest `
--query "Parameters[].Name"
\end{verbatim}

The ``standard'' linux image we will use is
\texttt{amzn2-ami-hvm-x86\_64-gp2}.

\subsection{Launching by name}\label{launching-by-name}

We can use Systems Manager to launch an instance using the following
syntax. Instead of giving an AMI directly we use \texttt{resolve:ssm:}
to tell AWS to look this value up in SSM.

\begin{verbatim}
aws ec2 run-instances `
 --subnet-id subnet-08f1cd51b749c3ce2 `
--image-id resolve:ssm:/aws/service/ami-amazon-linux-latest/amzn2-ami-hvm-x86_64-gp2 `
--instance-type t2.micro `
--key-name LAB_KEY `
--security-group-ids sg-0ea635cfaa7ab103e
\end{verbatim}

\subsection{Instance information}\label{instance-information}

Launching an instance will give a very long JSON in the format:

\begin{verbatim}
{
    "Groups": [],
    "Instances": [
        {
            "AmiLaunchIndex": 0,
            "ImageId": "ami-0bb3fad3c0286ebd5",
            "InstanceId": "i-04acdf703ecc03471",    "OwnerId": "637116340434",
            "...": "...",
            "ReservationId": "r-0536a89f39c4c6324"
}
\end{verbatim}

The public key is copied by AWS into the \texttt{ec2-user}'s
\texttt{authorized\_keys} file. This uses a combination of
\texttt{cloud-init} on the EC2 instance and the AWS Instance Meta Data
Service (IMDS), both of which we will meet later on.

\section{Connecting to instance}\label{sec:connecting-to-instance}

\subsection{Finding instance public
IP}\label{sec:finding-instance-public-ip}

Once our instance is Running, it will have a public IPv4 that AWS
transparently routes through to its private IP using NAT.

To find this out we can look it up in the console, or use the following
commands:

\begin{verbatim}
aws ec2 describe-instances --instance-id i-07affd402ac286fe6
\end{verbatim}

The public IP is listed with the network interface.

We can make some sense of this using PowerShell.

\begin{verbatim}
# First we capture the JSON and convert to PowerShell objects.
$reservations = $(aws ec2 describe-instances 
--instance-id i-07affd402ac286fe6 | ConvertFrom-Json)
# Then we extract the public IP.
$publicIp = $reservations.Reservations.Instances[0].NetworkInterfaces[0].Association.publicIp
\end{verbatim}

\subsection{Connecting to instance over
SSH}\label{sec:connecting-to-instance-over-ssh}

In PowerShell/Bash we can just use the SSH command to connect to the
instance. We will connect as the \texttt{ec2-user}. By default, SSH will
try all private keys so we don't need to specify which.

\begin{verbatim}
ssh ec2-user@$publicIp 
\end{verbatim}

The first time you connect to an instance you'll get a warning:

\begin{verbatim}
The authenticity of host '54.78.220.233 (54.78.220.233)' can't be established.
ECDSA key fingerprint is SHA256:8omkD5RLibZNgJJ/B7MAnL7IbEcrmCmIWFdQXbjJf60.
Are you sure you want to continue connecting (yes/no)?
\end{verbatim}

Just type \texttt{yes} here. Your local SSH client is just confirming it
hasn't seen this machine before. If a different key fingerprint shows
for the same IP you'll get a warning, which means a machine has been
changed for another.

If you see something like the following then you're connected:

\begin{verbatim}
       __|  __|_  )
       _|  (     /   Amazon Linux 2 AMI
      ___|\___|___|

https://aws.amazon.com/amazon-linux-2/
2 package(s) needed for security, out of 13 available
Run "sudo yum update" to apply all updates.
[ec2-user@ip-10-0-1-80 ~]$ 
\end{verbatim}

\section{EC2 instance termination}\label{ec2-instance-termination}

Instances can be terminated when no longer needed using the command:

\begin{verbatim}
aws ec2 terminate-instances --instance-ids i-07affd402ac286fe6
\end{verbatim}
