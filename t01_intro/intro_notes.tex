\chapter{Introduction}
\label{ch:intro}

\section{What's it all about?}
\label{sec:whats-it-all-about}

Cloud computing is essentially \textbf{USING SOMEONE ELSE'S COMPUTER!}.

The customer rents computing resources as needed to
build-up systems in a quick self-service way. Think of it as:
\begin{itemize}
\item booking a ryanair flight (vs buying your own plane)
\item renting a car in a foreign country (vs buying one) or
\item taking an AirBNB (vs buying a holiday home).
\end{itemize}

Like travel planning, learning how to design, provision and utilise
cloud computing resources is a learned skill developed through practice.

\section{Backend}
\label{sec:backend}

In this module we will primarily focus on backend processing, as opposed to frontend computing (desktops, mobile devices).
As with any computing system, backend processing generally involves three key ingredients:

\begin{description}
\item[Compute power:]
in the form of CPU cycles in server hardware, possibly shared by a
hypervisor and managed by an operating system.
\item[Storage:]
in the form of magentic disks, SSDs, tape and other media that is
directly attached to server hardware or connected over a network.
\item[Connectivity:]
so that the remote computing resource can be utilised by its intended
consumers.
\end{description}

The earliest large computers (so-called mainframes like IBM System/360)
led to smaller minicomputers (DEC PDP 8/11, VAX, IBM AS/400) and later
on to PC-based servers running Linux / UNIX / Windows. Yet many of the
ideas we'll see in Cloud particular around pricing and pay-per-use
actually were and are evident in Mainframes.

\subsection{Batch processing}
\label{sec:batch-processing}

Historically a lot of backend processing was batch-oriented. This is
still encountered in business processes (e.g.~a bank processes payments
overnight or utility sends a bill once a month).

\subsection{Services}
\label{sec:services}

The alternative to batch processing is service-based processing, where a
server allows one or more clients to connect and processes data in real
time. Services commonly encountered include:

\begin{itemize}
\item
  \textbf{On the public internet} are web servers (HTTP, HTTPS), file
  transfer protocol (FTP), domain name servers (DNS), remote shell login
  (SSH), remote desktop (RDP), network time protocol (NTP), remote sync
  (rsync), virtual private network (VPN).
\item
  \textbf{Within private networks:} all of the above plus file sharing
  (NFS, CIFS/SMB, AFS), database servers (MySQL, PostgreSQL, Oracle
  etc), message brokers (ActiveMQ, RabbitMQ), media sharing (UPnP/DLNA),
  block storage (iSCSI), mainframe access (3270, 5250 emulation),
  insecure remote shell login (telnet, rsh) and custom servers. Many of
  these services in theory could run over the internet but for security,
  performance and other reasons they generally don't.
\end{itemize}

We will assume in this module that all services communicate over
standard TCP/IP. For a given service, the client and server implement a
common protocol that normally has an associated standard port number.

\subsection{Building systems}
\label{sec:building-systems}

It is important to remember in practice that a service may be a client
of another service, and that most systems of even modest complexity
involve multiple interdependent services:

\begin{itemize}
\item
  A web application (PHP on Apache/mod\_php or Spring with Tomcat) might
  access a database (Oracle), a messaging queue (ActiveMQ) and an
  in-memory cache (Redis).
\item
  A file server running Windows Server provides SMB/CIFS file shares
  that are on a drive provisioned on an iSCSI volume shared from a
  FreeBSD UNIX machine.
\end{itemize}

\section{Service requirements}
\label{sec:service-models}

Assuming we are dealing with backend server environments, our services
are assumed to consist of software running on appropriate hardware.
Server hardware requires power, connectivity and cooling in a secure and
fireproof environment with appropriate channels to manage the hardware
remotely. Normally we require resiliency in the infrastructural services
and connectivity so that the service remains available. We can first
distinguish among a number of common patterns for where our services are
hosted:

\begin{description}
\item[On-site]
  within a suitably equipped data centre environment providing power,
  connectivity, cooling and security with the appropriate resiliency for
  the required service availability.
\item[Co-located]
  where our hardware is located in a data centre facility that provides
  space, power, connectivity, cooling and security with the appropriate
  levels of resiliency.
\item[Hosted]
  where a provider provisions equipment and software that we require.
  
  \begin{itemize}
  \item
    Hosted environments tend to involve manual setup and can provide both
    bare-metal servers, managed physical servers and virtual servers.
  \item
    Hosting provider will themselves be provisioning onsite or co-lo, or
    may be reselling from elsewhere!
  \item
    Many co-lo data centres also offer hosting as a service.
  \end{itemize}
\end{description}

All of these environments require significant upfront \emph{capital
expenditure}, must be right-sized from the beginning and scaling
requires significant effort.

\section{Expenditure}
\label{sec:expenditure}

Setting up and running any service requires expenditure, which can be
broken down into:

\begin{description}
\item[Capital expenditure (CAPEX):]
money spent on goods/services necessary for future performance. Usually
fixed / non-consumable assets. CAPEX is often made using borrowed money.
\item[Operational expenditure (OPEX):]
money spent on goods/services necessary for day-to-day running of the
business. Often supplies/inputs into output. Needs to be self-financing
if the service/business is to survive.
\end{description}

Note that whether a particular item is CAPEX or OPEX depends on the
context. To a homeowner, a bale of concrete blocks for an extension
represents CAPEX. To the shop supplying them, they are an OPEX to buy.

\textbf{Key advantage} of cloud computing is shifting CAPEX to OPEX.

\section{Cloud definition and characteristics}
\label{sec:essential-characteristics}

The term ``cloud'' primarily appeared in its current context when AWS
released their so-called Elastic Compute Cloud in 2006. Rather than an
explicit definition, NIST describes five essential characteristics for a
particular service to be termed a ``cloud service'':

\begin{description}
\item[On-demand self-service]
allowing consumer to provision required services without human
interaction.
\item[Broad network access]
allowing consumer to access provisioned services over standard network
interfaces (e.g.~TCP/IP over internet).
\item[Resource pooling]
where provider uses a large pool of resources to serve consumers
on-demand, without the consumer having any affinity to a physical
resource.
\item[Rapid elasticity]
to enable a consumer to vertically/horizontally scale their provisioned
capacity.
\item[Measured service]
where a consumer pays for provisioned/used services at a relatively
granular level, subject to resource limits.
\end{description}

If a service broadly meets these, then it can be termed a cloud service.
They key differentiating factors are the self-service nature, the
locatation independence, the seemingly unlimited pool of resources on
the backend and the ability to control and pay for what you use on a
short-term basis.

\section{Main providers}
\label{sec:main-providers}

Main cloud providers: AWS, Azure, Google, IBM, Oracle \& more, DigitalOcean. 


\section{Cloud service models}
\label{sec:cloud-service-models}

Within the cloud NIST describes three different service models:

\begin{description}

\item[Infrastructure as a Service (Iaas):]
provider makes available compute power in the form of virtual machines,
storage in multiple forms and allows networks to be constructed to link
these resources together. Obviously you don't have access to the
hardware.

\begin{itemize}

\item
  \textbf{How to identify:} you are dealing with CPU / RAM / storage
  specifications, operating systems (Windows / Linux), networking (IP
  addressing, subnet masks).
\end{itemize}

\item[Platform as a Service (PaaS):]
components such as database servers, integration components (queues,
notification systems), object storage that can be provisioned directly
from the web / CLI.

\begin{itemize}
\item
  \textbf{How to identify:} you're not directly interacting with CPU,
  RAM, storage, OS, networking.
\end{itemize}

\item[Software as a Service (SaaS):]
provider makes software available for use over standard network
protocols (HTTP/RDP/SSH) either by humans or an API (e.g.~Github).
\end{description}

In reality, the lines between IaaS, PaaS and SaaS are often blurred.
They are probably better thought of as a continuum from SaaS through PaaS to IaaS.
It can be difficult to place many cloud services exclusively into one of these three categories, nor is it necessary to.
Many solutions built including cloud technologies will include all three.


\section{Interface}
\label{sec:interface}

Most cloud providers support a number of channels for provisioning resources and interacting with them:

\begin{description}
\item[Web console]
providing manual point-and-click interaction

\begin{itemize}

\item
  Easy to explore, BUT not a good idea to use exclusively long-term.
\end{itemize}
\item[Command-line interface (CLI)]
command can be installed on your own computer and can be run from
PowerShell / Bash

\begin{itemize}
\item
  It can be tempting to avoid the CLI but this is generally a foolish
  decision in terms of time, precision and opportunities to automate!
\item
  The CLI commands for most providers (e.g. \texttt{aws},
  \texttt{gcloud}) provide an extensive online help system with
  examples. For AWS try typing \texttt{aws\ help}.
\item
  CLI avoids cumbersome manual login process - another reason to use it.
\item
  The CLI is available from the web console in AWS, Google \& others.
\end{itemize}

\item[Software Development Kit (SDK):]
for applications running on cloud and elsewhere (local desktop, mobile)
that need to interact with cloud resources.

\begin{itemize}
\item
  We won't spend much time with SDKs but may encounter them later on!
\item
  Example: a Python script needs to push a message to a messaging queue
  provisioned on AWS SQS.
\end{itemize}
\end{description}



\section{Service agreement}\label{service-agreement}

Cloud Providers will offer their services as part of a service
agreement, which specifies a number of promises, limitations and
obligations:

\begin{description}
\item[Promises]
are guarantees by the cloud provider of things it will do for the
customer.
\item[Limitations]
limit the scope of the promises that a cloud provider makes to a
customer.
\item[Obligations]
specify rules that a customer must comply with when using the cloud
service.
\end{description}

\section{Costs}
\label{sec:costs}

Metered costs for cloud usage depend on the particular service under
discussion. Broadly, these will break down into:

\begin{table}
  \begin{tabularx}{1.0\linewidth}{l l X}
    \toprule
    \textbf{Charge type} & \textbf{Common units} & \textbf{Remarks}\\
    \midrule
    Setup/teardown & Per resource unit & ~ \\
    Time-based & Days / hours / mins / seconds & Often for IaaS components\\
    Usage-based & Requests / invocations & Often for PaaS components\\
    Data Storage & MB / GB / TB etc & Data at rest\\
    Data Transfer & MB / GB / TB etc & Inbound, outbound, internal may differ\\
    \bottomrule
  \end{tabularx}
  \caption{Cloud charge types}
  \label{tab:cloud-charge-types}
\end{table}



\section{Tips}
\label{sec:tips}

\begin{itemize}
\item
  \textbf{Ignore ill-informed hype} as much as possible. Many people
  commenting on cloud (and other) technological matters are no more
  informed that a random person on the street. Always refer to multiple
  sources, provider manuals and your own experience.
\item
  Cloud vs.~traditional non-cloud is \textbf{not inherently good or bad
  in itself}. The \emph{business need} and \emph{technical suitability}
  should always dictate if a cloud-based or onsite/co-located/hosted
  service is the right choice.
\item
  Many \textbf{systems will have both cloud and non-cloud components}.
  Hybrid solutions where a cloud-based copmponent communicates with an
  onsite component often provide the optimal solution.
\item
  There is often \textbf{more than one right answer} to a particular
  problem. The preferred solution might not always be the most
  technically ``correct''.
\item
  Learn and use \textbf{Command-Line Interfaces (CLIs)} for everything
  you use where possible. Not just cloud infrastructure but everything!
\item
  \textbf{Automation is key}. You NEED to be able to script repetitive
  operations so that you don't waste time on them, and that they're done
  correctly, on schedule, every time.
\end{itemize}

