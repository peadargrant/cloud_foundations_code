\chapter{Infrastructure groups}
\label{ch:infrastructure-groups}


The major ``full service'' cloud providers separate their infrastructure
geographically. This is done for two reasons. Firstly it lets cloud
providers separate services in different geographical regions from each
other for legal and business reasons. Secondly it facilitates redundancy
in case of failures.

\section{Regions}
\label{sec:regions}

\textbf{Regions} are \textbf{designed to separate data and resources} for business and legal reasons.

\begin{itemize}
\item
  Regions function independently of each other. Essentially they are
  different instances of the cloud provider's infrastructure.
\item
  Regions usually chosen such that legal environment is similar across
  them: e.g.~varied FOI, GDPR, HIPAA, COPPA, data siphoning regulations.
\item
  Provider may introduce and update services to different regions at
  different times.
\item
  Pricing may vary across regions for the same services.
\item
  Most cloud providers identify each region with both a human readable
  name like ``EU (Ireland)'' and a code name like \texttt{eu-west-1}. We
  will in class use only the code names for the most part.
\item
  \textbf{Must consider region when deploying most services!}
\end{itemize}
Also need to consider:
\begin{description}
\item[Non-geographic regions]
are purpose-specific rather than geographical, like AWS GovCloud.
\item[Geography]
is a region-grouping that Azure use. AWS doesn't have a multi-region
grouping concept.
\end{description}
\begin{itemize}
\item
  Legal environment is normally consistent across all regions in a
  Geography.
\end{itemize}

\subsection{Region listing}
A list of regions can be accessed in the CLI using
\begin{verbatim}
aws ec2 describe-regions
\end{verbatim}


\section{Availability zones}
\label{sec:availability-zones}

\textbf{Availability zones} are a \textbf{technical construct} to tolerate failures in physical
infrastructure.

\begin{itemize}
\item
  Each AZ normally consists of \(\ge 1\) data centres, each internally
  redundant. The data centres in an AZ are connected to each other by
  low-latency highly-redundant network links.
\item
  AZs are identified only by a code name derived from their region code
  name. For the \texttt{eu-west-1} region the corresponding AZs are
  \texttt{eu-west-1a}, \texttt{eu-west-1b} and \texttt{eu-west-1c}.
\item
  Note that the AZ code \texttt{a,\ b,\ c} mappings to physical AZs
  \emph{are not} consistent across different AWS accounts. This is to
  balance usage across AZs. (Many people will just choose the \texttt{a}
  AZ since it appears first in the list.)
\item
  \textbf{In general, we need to consider AZs when deploying IaaS
  services but NOT PaaS.}
\end{itemize}

\subsection{AZ listing}

A list of availability zones can be accessed in the command-line interface:
\begin{verbatim}
aws ec2 describe-availability-zones
\end{verbatim}

\section{Edge locations}
\label{sec:edge-locations}

\textbf{Edge locations} are an entirely parallel structure provididing access points and edge caching for users of customer applications.
\begin{itemize}
\item These are normally a separate set of locations to the provider's regions/AZs for customer use.
\item Only encountered when using certain services (not this semester).
\end{itemize}



\section{Infrastructure maps}
\label{sec:infrastructure-maps}

The providers all do a better job on constantly updating their global
infrastructure maps than any lecture notes will:

\begin{itemize}
\item
  \textbf{AWS:}
  \url{https://aws.amazon.com/about-aws/global-infrastructure/}
\item
  \textbf{Google Cloud:} has a similar breakdown of regions and AZs to
  AWS. \url{https://cloud.google.com/about/locations}
\item
  \textbf{Azure:}
  \url{https://azure.microsoft.com/en-us/global-infrastructure/}
\end{itemize}


