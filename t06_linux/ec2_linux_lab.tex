\documentclass{pgnotes}

\title{EC2 linux lab}

\begin{document}

\maketitle

\section{Create EC2 and VPC}\label{create-ec2-and-vpc}

Go into the console and delete any existing EC2 and VPCs you have
created.

Create an EC2 instance following last week's lab in the AWS Console.

Use the \texttt{./check\_lab\_vpc.ps1} script to confirm that your VPC is correct before proceeding to create the EC2.


\section{Linux familiarisation}\label{linux-familiarisation}

These tasks are to re-familisarise you with Linux.
You may have to research some of these commands. 
Attempt these while logged in to Amazon Linux.

\subsection{Yum}
\label{yum}

Use \texttt{yum} to update all software

\subsection{Git}
\label{git}

Use \texttt{yum} to install git.

Try to checkout this repository on the EC2 instance.

\subsection{Text editor}\label{text-editor}

Use the \texttt{nano} text editor.

\subsection{Text-mode web browser}

Install the elinks web browser using yum. 

\subsection{Web server}\label{web-server}

Install apache web server:

\begin{verbatim}
yum -y install httpd
\end{verbatim}

Enable to start

\begin{verbatim}
sudo systemctl enable httpd
\end{verbatim}

Start now

\begin{verbatim}
sudo systemctl start httpd
\end{verbatim}

Try to connect to your instance's public IP from your browser over HTTP
(not HTTPS!).

You will find it won't work. This is because security group is not
letting traffic in.

Modify your security group in the console to permit web traffic in. 


\subsection{Website}\label{website}

Navigate to the \texttt{/var/www/html} directory.
Create an \texttt{index.html} using \texttt{nano}.
Refresh your web browser and confirm that your page is now online. 


\section{Cleanup}

Make sure to terminate your instance and delete your VPC when done.

\end{document}

