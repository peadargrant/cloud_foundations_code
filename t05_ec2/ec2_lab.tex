\chapter{EC2 lab}

\section{Check your setup}

Before starting, open PowerShell/Bash and run:
\begin{minted}{powershell}
# change to where you've cloned the folder
cd ~/Desktop/cloud_foundations

# run the lab checks script:
./lab_checks.ps1
# fix any issues identified before continuing

# then change to the ec2 topic folder
cd *_ec2
\end{minted}

\section{VPC creation}

\subsection{Check script}

After each step you should run the check script and confirm that the step has been done correctly.
\begin{verbatim}
./check_lab_vpc.ps1
\end{verbatim}

\subsection{Creation steps}

\begin{enumerate}

\item 
  Create the VPC named \texttt{LAB\_VPC} with IP range 10.0.0.0/16.

\item
  Create the Subnet named \texttt{LAB\_SUBNET\_1} with:
[6~  \begin{description}
  \item[IP range] 10.0.1.0/24
  \item[automatic public IP assignment] turned on
  \end{description}
  
\item
  Create the Internet Gateway named \texttt{LAB\_GATEWAY}.

\item
  Attach the Internet Gateway to the VPC.

\item
  Find the route table for the VPC.
  Add a route to send traffic for anywhere (0.0.0.0/0) to your internet gateway.

\item
  Create the Security Group named \texttt{LAB\_SG} with description \texttt{LAB\_SG}.

\item 
  Add a rule to your security group permit SSH traffic (TCP port 22) inbound from anywhere on the internet (0.0.0.0/0).
  
\end{enumerate}

\section{EC2 instance}

Create EC2 instance as follows:
\begin{enumerate}
\item Open the instances page by browsing to Services, EC2, instances.

\item Hit Launch Instances.

\item Name the instance \texttt{LAB\_INSTANCE}.

\item Pick Amazon Linux from the AMI list.

\item Check that instance type is \texttt{t2.micro} (should be default)

\item Under Key Pair drop down to \texttt{MAIN\_KEY}.

\item On Network Settings hit Edit. Choose your \texttt{LAB\_VPC} from the list.

\item Ensure that your \texttt{LAB\_SUBNET\_1} is selected.

\item Ensure that Auto-assign public IP is set to Enable.
  
\item Choose Select existing security group.

\item From the list pick \texttt{LAB\_SG}.

\item Hit Launch Instance in the bottom right.

\item Click the Instances link and notice that your instance will be Pending before changing to Running.
  
\end{enumerate}

\section{SSH to your instance}

\begin{enumerate}
\item Click into your instance in the list.

\item Click the copy button beside Public IPv4 address.

\item In PowerShell/Bash type:
\begin{verbatim}
ssh ec2-user@<Public IP here>
# example:
ssh ec2-user@3.251.82.154
\end{verbatim}

\item
The first time you connect to a host you'll get a warning:

\begin{verbatim}
The authenticity of host '54.78.220.233 (54.78.220.233)' can't be established.
ECDSA key fingerprint is SHA256:8omkD5RLibZNgJJ/B7MAnL7IbEcrmCmIWFdQXbjJf60.
Are you sure you want to continue connecting (yes/no)?
\end{verbatim}

Just type \texttt{yes} here.

\item 
If you see something like the following then you're connected:

\begin{verbatim}
       __|  __|_  )
       _|  (     /   Amazon Linux 2 AMI
      ___|\___|___|
\end{verbatim}

\end{enumerate}

\section{Cleanup}

Delete all resources you made in this lab:
\begin{enumerate}
\item Terminate the instance.
\item Delete the VPC.
\end{enumerate}

