\chapter{Command Line Interface Setup}
\label{ch:cli-setup}

\section{Using your own computer}

If you're on a lab computer then skip ahead to \autoref{sec:powershell-execution-policy}.
I only support the lab configuration but you are welcome to try out and use your own machine (incl. MacOS, Linux etc) if you can support it yourself.
All of the software used here is cross-platform.

Mac, Linux, Unix users will have no problems installing the AWS CLI, and the commands work identically to those on Windows.


\subsection{Software installation}

Required software:
\begin{description}
\item[PowerShell] on non-MS systems\\from \url{https://github.com/PowerShell/PowerShell/releases/tag/v7.2.6}
\item[Git] from \url{https://git-scm.com/downloads}
\item[AWS CLI] from \url{https://aws.amazon.com/cli/}
\end{description}

\subsection{PowerShell usage}

The difference between Mac/Linux and Windows centres on the use of
Bash/zsh by Mac/Linux/UNIX vs PowerShell on Windows. The AWS CLI is
perfectly scriptable using Bash, particularly in conjunction with
\texttt{jq} to parse JSON. However, some of the scripts you will be
provided with in this module will be PowerShell only due to time
constraints.

The good news is that PowerShell Core 7 can be installed easily on a Mac
with no issues. You \emph{do not} need a Windows VM on your Mac to use
any of the PowerShell or AWS commands in \emph{this} course. Please go
to the \href{https://github.com/PowerShell/PowerShell}{PowerShell page
on GitHub} for instructions.

When you have PowerShell installed, open the Terminal app and type
\texttt{pwsh} and you'll be at a PowerShell prompt.

\section{PowerShell execution policy}
\label{sec:powershell-execution-policy}

(Not needed on non-Windows systems.)
We will use a number of PowerShell scripts throughout the module. 
PowerShell by default will not allow scripts to run that were downloaded from online.
Run the following command to change this behaviour:

\begin{verbatim}
Set-ExecutionPolicy -ExecutionPolicy RemoteSigned -Scope CurrentUser
\end{verbatim}

Answer \textit{all} when prompted.


\section{Setup AWS command-line interface}
\label{sec:setup-aws-command-line-interface}

\begin{enumerate}
\item Log in to the AWS CLI.
\item Click your name in the top right and then Security Credentials.
\item Scroll down to the Access Keys section and hit Create access key.
\item For now check the box beside \textit{I understand creating a root access key is not a best practice, but I still want to create one} prompt and hit Create access key.
\item In PowerShell run:
\begin{verbatim}
aws configure
\end{verbatim}
It will prompt first for an access key. 
\item Copy the Access key from your web browser using the copy button to the left. Paste into Powershell (right click pastes). Press enter. You should see a prompt for the secret key.
\item In your web browser press Show beside Secret access key.
\item Click the copy button left of the secret access key. Paste into Powershell (right click pastes). Press enter. You should see a prompt for default region name.
\item Type \texttt{eu-west-1} and press enter.
\item At the next prompt for output format, just press enter again.
\end{enumerate}





