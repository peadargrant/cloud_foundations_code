\chapter{EC2 CLI (Windows)}

Setting up Windows EC2 instance is very similar to that of a linux instance.

\section{Windows Image IDs}

We can use Systems Manager to launch an instance using the following
syntax. Instead of giving an AMI directly we use \texttt{resolve:ssm:}
to tell AWS to look this value up in SSM.

\begin{minted}{powershell}
# Look up available windows AMIs
aws ssm get-parameters-by-path `
--path /aws/service/ami-windows-latest `
--query "Parameters[].Name"

# run instance w/ specified Windows AMI
aws ec2 run-instances `
--subnet-id $SubnetId `
--image-id resolve:ssm:/aws/service/ami-windows-latest/Windows_Server-2022-English-Full-Base `
--instance-type t2.micro `
--key-name MAIN_KEY `
--security-group-ids $GroupId
\end{minted}

\section{Windows Password Data}
\label{sec:windows-password-data}

See:\\
\url{https://docs.aws.amazon.com/cli/latest/reference/ec2/get-password-data.html}

\begin{minted}{powershell}
# Assume we have the instance id stored in $InstanceId

aws ec2 get-password-data `
--instance-id $InstanceId `
--priv-launch-key file://~/.ssh/id_rsa
\end{minted}

\section{Lab exercise}

Make sure you can login to the AWS Console.
Then use the CLI to do the following:
\begin{enumerate}
\item Create a VPC \texttt{LAB\_VPC}, IP range 10.0.0.0/16.
\item Create a Subnet \texttt{LAB\_SUBNET\_1} within your VPC, IP range 10.0.1.0/24.
\item Turn on auto IP address assignment in the subnet.
\item Create an Internet Gateway \texttt{LAB\_GATEWAY}.
\item Attach the internet gateway to your VPC.
\item Alter the route table to send traffic for anywhere \texttt{0.0.0.0/0} to the internet gateway.
\item Create a security group \texttt{LAB\_GROUP}.
\item Modify the security group to permit traffic inbound on SSH (22) and RDP (3389) protocols from anywhere \texttt{0.0.0.0/0}.
\item Create an instance \texttt{LAB\_INSTANCE} to run Windows.
\item Wait until the instance is running.
\item Get the instance ID.
\item Get the instance's public IP.
\item Get the windows password data as shown in \autoref{sec:windows-password-data}.
\item Use RDP to login to the instance.
\end{enumerate}


