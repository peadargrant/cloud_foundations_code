\chapter{Notification Services: SNS}
\label{ch:sns}
  
\section*{Required reading}\label{required-reading}

\begin{enumerate}
\item
  \url{https://docs.aws.amazon.com/sns/latest/dg/welcome.html}
\item
  \url{https://docs.aws.amazon.com/general/latest/gr/aws-arns-and-namespaces.html}
\end{enumerate}

\section{Use cases}\label{use-cases}

\begin{itemize}
\item
  Sending end-user e-mails, push notifications and SMS messages to
  individual or groups of users.
\item
  Notifying or triggering actions when \textbf{events} occur in parts of
  a system.
\end{itemize}

\subsection{Asynchronous processes}\label{asynchronous-processes}

\begin{description}
\item[Synchronous]
process is where the requestor makes a request, waits for it to
complete, receives the results (if any) before continuing.
\item[Asynchronous]
process is where the requestor makes a request and continues on working
whilst its request is being processed. Results are returned either by
polling or callback.
\end{description}

\section{Simple notification service
(SNS)}\label{simple-notification-service-sns}

SNS is an \textbf{asynchronous} notification service. It is useful for
both system-to-system and system-to-user communications. \autoref{fig:sns-components} shows the key
components of SNS.

\autoimage{notification_system}{Notification system}{notification-system}

The key unit of SNS is the \textbf{topic} to which two types of client
connect:

\begin{description}
\item[Publisher]
wishes to send message to a number of subscribers via the topic. Any
number of publishers per topic (within limits).
\item[Subscribers]
wish to be notified of messages sent to the topic. SNS \textbf{pushes}
messages to subscribers. Subscriptions to a single topic can be in a
number of different forms. Any number of subscribers per topic (within
limits).
\end{description}

\subsection{Properties}\label{properties}

There are some important properties that distinguish SNS from queueing
systems (like SQS):

\begin{enumerate}
\def\labelenumi{\arabic{enumi}.}
\item
  Messages always attempted to be delivered to all topic subscribers
  (fan-out, ).
\item
  Topic subscribers are inhomogeneous (not all the same) and may process
  received message in different ways.
\item
  No persistence. Message received by whatever subscribers are present
  when message arrives. No ``catch-up''.
\end{enumerate}

\subsection{Topic creation}\label{topic-creation}

\begin{verbatim}
# create topic
aws sns create-topic --name my-topic

# or, better:
$TopicArn=(aws sns create-topic --name my-topic | ConvertFrom-Json).TopicArn
\end{verbatim}

\subsection{Naming}\label{naming}

Amazon Resource Names (ARNs) are the canonical naming format for SNS
topics. Format:

\begin{verbatim}
arn:aws:sns:eu-west-1:637146340431:my-topic
|   |   |   |         |            |
1   2   3   4         5            6
\end{verbatim}

Components:

\begin{enumerate}
\item
  All ARNs begin with \texttt{arn}.
\item
  \textbf{Partition} always \texttt{aws} unless \texttt{aws-cn} (China)
  or \texttt{aws-us-gov} GovCloud.
\item
  \textbf{Service} here \texttt{sns}
\item
  \textbf{Region} as per your default
\item
  \textbf{Account ID} number
\item
  \textbf{Topic name} as entered
\end{enumerate}

If you have the required information you can construct the ARN for any
SNS topic.

\subsection{Listing topics}\label{listing-topics}

\begin{verbatim}
# list of topic ARNs
aws sns list-topics

# list of topic ARNs as PowerShell list
$Topics=(aws sns list-topics | ConvertFrom-Json).Topics
\end{verbatim}

\subsection{Deleting topics}\label{deleting-topics}

\begin{verbatim}
# assume $TopicArn holds the topic ARN

aws sns delete-topic --topic-arn $TopicArn
\end{verbatim}

\subsection{SNS clients}\label{sns-clients}

SNS clients come in the form of publishers and subscribers, .

We will first meet subscribers and then publishers.

\section{Subscribers}\label{subscribers}

Subscribers to a topic can be managed by the \texttt{subscribe} and
\texttt{unsubscribe} commands. Different types of subscribers possible
to a topic:

\autoimage{sns_clients}{SNS clients}{sns-clients}


\begin{description}
\item[Machine-to-human:]
email, sms
\item[Machine-to-machine (generic):]
email-json, http, https
\item[Other AWS services:]
application, sqs, lambda
\end{description}

\subsection{Subscribing}\label{subscribing}

Subscribing requires a topic ARN, protocol and usually an endpoint.

\begin{verbatim}
# full info on command options
aws sns subscribe help

# assume $TopicArn holds ARN for topic

# subscribe to e-mail 
aws sns subscribe `
--topic-arn $TopicArn `
--protocol email ` # also try email-json to see difference
--notification-endpoint "mail@mydomain.ie"
--return-subscription-arn
# may be a confirmation step required

# see help for other formats
\end{verbatim}

\subsection{Subscription ARN}\label{subscription-arn}

Each subscription itself has an ARN, based on the topic ARN. Example:

\begin{verbatim}
arn:aws:sns:eu-west-1:637116340434:my-topic:74b543bc-0eab-46ea-81c1-0a654a6fb236
1   2   3   4         5            6        7
\end{verbatim}

The first 6 components are identical to the topic ARN. The last
component (7) identifies the subscription.

\subsection{Checking if confirmed}\label{checking-if-confirmed}

\begin{verbatim}
aws sns get-subscription-attributes  --subscription-arn $SubscriptionArn
\end{verbatim}

Check output of \texttt{get-subscription-attributes} for
\texttt{PendingConfirmation}.

\subsection{Unsubscribing}\label{unsubscribing}

Unsubscribing requires the subscription ARN only:

\begin{verbatim}
aws sns unsubscribe --subscription-arn $SubscriptionArn
\end{verbatim}

\section{Publishing}\label{publishing}

Messages are \emph{published} to a specific topic and are delivered to
subscribers.

\begin{verbatim}
# send message to topic 
aws sns publish --message "hello there" --topic-arn $TopicArn
\end{verbatim}


\section{SNS Exercise}
\label{sec:sns-exercise}

For the following exercise, attempt to do as much as possible using the command-line interface.
Use the inbuilt help to discover the required commands.
Document the required commands in your notes for each step.

\begin{enumerate}
\item
  Create an SNS topic named \texttt{lab-sns-test}.
\item
  Add two e-mail subscribers (e.g.~your student and personal email).

  \begin{itemize}
  \item
    Notice how SNS deals with subscription confirmation! Confirm
    \emph{one} of the subscriptions only for now.
  \end{itemize}
\item
  Send a test message to the topic.

  \begin{itemize}
  \item
    Is it received by all endpoints?
  \item
    Confirm the other subscription.
  \item
    Does the message now send to the second subscription? Does your
    finding agree with the documentation?
  \item
    Now try a new test message (with different content) and not what
    happens.
  \end{itemize}
\end{enumerate}


